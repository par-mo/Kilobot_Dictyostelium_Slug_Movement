 
 \documentclass[11pt,a4paper]{article}

\usepackage[a4paper,bindingoffset=0.2in,%
            left=.6in,right=.7 in,top=.9in,bottom=.6in,%
            footskip=.25in]{geometry}
 
\usepackage{graphicx}
 \usepackage{amssymb}
 
\usepackage{lineno}
%\addbibresource{mybibliography.bib}

%\journal{Journal Name}
\begin{document}
%% Title, authors and addresses
\begin{center}
{\fontfamily{ppl}\selectfont{\large\textbf{Self-organized Simulation Model for {\textit{Dictyostelium Discoideum}}\\ Slugs Movement with Kilobots}} }
\vspace{2mm}

{\fontfamily{qbk}\selectfont\textit{Mohammad Parhizkar \\ Giovanna Di Marzo Serugendo \\ \tiny{February 2019} }}


%\vspace{2mm}
%\small{Information Systems\\Geneva School of Social Sciences \\Centre Universitaire d'Informatique\\\vspace{3mm}
%{\fontfamily{phv}\selectfont\small{Superviser: Prof. Giovanna Di Marzo Serugendo}}}
\end{center}
%% Text of abstract
\begin{abstract}
Understanding the collective behaviors in nature and its potential links to engineering the collective artificial behaviors in swarm robotics have attracted the attention among researchers. They have various impacts on different domains such as cell-biology, cancer study, the swarm of drones and unmanned robots. Since the cancer cells share similar collective behaviors, the biomedicine researchers look into different examples from nature to design anti-cancer drugs to shrink tumors in human bodies. An exciting form of collective system is demonstrated by {\textit{Dictyostelium discoideum}}. 

   \end{abstract}

{\footnotesize\textit{Keywords:} {\textit{Dictyostelium discoideum}} - Kilobot - Swarm robotics - Multi-agent systems - Self-organising systems}

{\begin{center}\noindent\rule{14cm}{0.4pt}\end{center}}
%\noindent\makebox[\linewidth]{\rule{\paperwidth}{0.2pt}}
 %% ------------------------------------------------The first section ------------------------------------------------
\section{Introduction}
The idea of simulation of slug movement in \textit{Dictyostelium discoideum} with kilobots is to have a swarm of kilobots which moves in one direction in a decentralized way, in a leader-follower manner. These robots act independently and asynchronously in the environment. They keep a fair distance autonomously with their leader and their follower. With this technique, they retain the shape of the queue.  

Each slug is about $2-–4 mm$, and composed of up to $100k$ starved cells. It is capable of movement by producing a cellulose sheath around it. The slug attractants to light, heat, and humidity in a forward-only direction. Cyclic AMP and a substance called differentiation-inducing factor, help to form different cell types. In the aggregation phase of \textit{D. discoideum} cells differentiated into prestalk(pst) which is $20\%$ of the \textit{D.~discoideum} population and prespore(psp) which is $80\%$ of the cells population. The pst cells move to the anterior and psp cells move to the posterior end of the slug.

We design a self-organized model of small kilobots with minimum capabilities to move to like a chain. This group of robots reaches to keep the structure of the chain by local interactions only and by utilizing positive feedback.   In this article, five packs of 10 Kilobots are utilized to implement an inspiring model of \textit{Dictyostelium discoideum} and present the first-order and second-order collective behaviors of self-organized agents.  

In the desired chain, each kilobot needs to recognize the robot in front of it and the one behind it. This identification process happens in every time step. Interestingly, in this model we do not affiliate any id to the robots. Thus, they can move their local position in the chain intentionally. Since the model is decentralized, the position of the specific kilobot is represented by its distance to its leader and the distance to its follower. ..

In most of the experiments such as the Harvard experiment, a sequential ID  is allocated to each Kilobot. This ID number provide the knowledge to the robots to be aware of their positions based on the robot in front or and the robot in behind. In our model, Kilobots do not possess any specified ID number and elect their leader and follower in each iteration based on the nearest- neighbor distance \textit{(see Figure \ref{fig:propagation})}. 


In our model Kilobots operate in two states, pause or move, which is determined by the current state of their neighbors Kilobots in the chain (the leader and the follower). 

In this paper, we study the ...
%%------------------------------------------------------------------------------------------------------------
\section {Kilobots} 
Kilobot possesses an Atmega328 processor controller with 32KB of memory, which runs at 8MHz. To program Kilobots \cite{Nagpal2012}, we use an overhead infrared transmitter instead of plug-in cable, which allows us to program all the Kilobots collectively once in a time.

Kilobots utilize vibration motors for locomotion, a reflective infra-red LED, and distance sensing to communicate with other robots in their neighborhoods. Communications are transmitted by pulsing message in a range of \textit{10cm}  \cite{Nagpal2012}. All the neighbors in the range from any direction can receive the emitted light, which has been reflected by the surface. 

Sensing measured distances between neighbors as location feedback helps the robot movement to be relatively correct. This feature is useful to advance the practice of collective behaviors. However, the lack of bearing system makes some challenges in the swarm coordination implementation \cite{griffith2016evolutionary}. 




\subsection{Direction Finding}
A Kilobot can determine the distance to the message sender based on the signal strength but not the direction change. However, the infra-red receiver assists robots to continuously collect and observe the signal strength in order to determine the neighbor pose \cite{griffith2016evolutionary}.
\subsection{Ad-hoc counting system}
Each robot can choose just one leader and one follower based on the ad-hoc gradient model. 
\subsection{Leader-follower behaviors} This behavior is used in keeping the chain pattern. Where the leader kilobot decides to develop a path and the kilobots behind follow in relatively the same direction. A Kilobot in the pause state will wait stable until both Kilobots in front and behind have settled and transferred to the pause state. Next, the Kilobot can change to the moving state.
The Kilobots in moving state proceed their direction based on the acquired strength message from their leaders \cite{Beckerleg2016EvolvingRobot}. Each Kilobot transfers again to the pause state when the distance to its relative leader is less than one cm. 
\subsection{Birds flock}

%%------------------------------------------------------------------------------------------------------------
\section {Chain Formation} 
In real \textit{D. discoideum}, cells have the capability of pulling and pushing each other, which is not possible with Kilobots. We define the Kilobot gradient, which has the same effect of cAMP molecules among the original cell. We choose a robot as the head of the slug. This robot is responsible for finding a good direction and for leading the whole slug. This robot acts as the root of the gradient. We call the head kilobot as the ``source''. Thus the gradient for the source is zero. The nearest robot after the source has the gradient one and the second nearest robot is two and so on. In each time-step, when the robots send the signals, they share the amount of their gradients with each other as well. Therefore, the robots can replace their positions in the chain. Utilizing this method supports the kilobots to recognize their leader and follower in each time-step. \textit{Figure \ref{fig:propagation}} shows the different distances between robots and their propagation areas. 


%%------------------------------------------------------------------------------------------------------------
\section{Collective Decision-Making Model}
We design a decentralized decision-making model to allow the robots to decide timely between different choices, based on their neighbors in the chain. In this model, each agent \footnote{each Kilobot} has the ability to discriminate between two options: 1. going back to the previous-state or 2. keep going with its current-state. 

%%------------------------------------------------------------------------------------------------------------
\section{Three Distance Categories}
We define three different thresholds for the distances to make the decision-making process easier for the robots. Each kilobot can send and receive the message from others in range of \textit{10 cm}. We defined three different thresholds based on the distance between each two robots. 

\begin{itemize}
    \item Short-distance \textit{(SD)}: distance between \textit{2 cm} to \textit{8 cm}
    \item Medium-distance \textit{(MD)}: distance between \textit{8 cm} to \textit{14 cm}
    \item Long-distance \textit{(LD)} : distance between \textit{14 cm} to \textit{20 cm}
    
\end{itemize}
 \begin{figure}[h]
     \centering
 \includegraphics[scale=0.4]{Figs/PropagationArea_BW.pdf}
     \caption{ The figure shows the distances between four different kilobots, which is \(d_{12}<d_{13}\). In this example, if the robot $R_1$ is the origin, its gradient is equal to zero.  Therefore, the gradient value of $R_2$ is equal to one, since it is the nearest neighbor for $R_1$. Then, the gradient value for the robot $R_3$ is equal to two, because it is the nearest neighbour $R_2$. Eventually, the gradient value for $R_4$ will be three.  }
     \label{fig:propagation}
  \end{figure}
%%------------------------------------------------------------------------------------------------------------
\section {Kilobot Finite Machine }
In each time-step, the movement of a kilobot is either `left' or `right' or `forward'. The global objective of the system is to keep the shape of the chain. Also, the minimum objective of each robot in every time-step is to get closer to its leader and to be close enough to its follower.  
\subsection{Nine different states}
\begin{itemize}
    \item Short-distance to the leader/ Short-distance to the follower (SDLSDF)   
    \item Short-distance to the leader/ Medium-distance to the follower (SDLMDF)
    \item Short-distance to the leader/ Long-distance to the follower (SDLLDF)   
    \item Medium-distance to the leader/ Short-distance to the follower  (MDLSDF) 
    \item Medium-distance to the leader/ Medium-distance to the follower  (MDLMDF) 
    \item Medium-distance to the leader/ Long-distance to the follower (MDLLDF) 
    \item Long-distance to the leader/ Short-distance to the follower (LDLSDF) 
    \item Long-distance to the leader/ Medium-distance to the follower (LDLMDF) 
    \item Long-distance to the leader/ Long-distance to the follower (LDLLDF) 
\end{itemize}
%%------------------------------------------------------------------------------------------------------------
 \section{Individual Decision Making Process }
 The movement of each kilobot in the chain is determined by its current state, its previous state and its distance to its leader and its follower in the chain. \textit{Figure \ref{fig:states}} consists of states and their relationships. The arrows show that if the robot is in this situation its next move will be on its previous state. Thus, it moves in opposite was as its previous movement direction. For the conditions, there is no line, it means the robot keeps its logic as it is without any change. 
 
 \begin{figure}[h]
     \centering
 \includegraphics[scale=0.7]{Figs/States.pdf}
     \caption{Nine different states for previous-state and current-state for each robot and its relationship.  }
     \label{fig:states}
  \end{figure}
 

\bibliographystyle{plain}
\bibliography{references.bib}

\end{document}

%%
%% End of file `elsarticle-template-1-num.tex'.