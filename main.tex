 
 \documentclass[10.5pt,a4paper]{article}

\usepackage[a4paper,bindingoffset=0.2in,%
            left=.6in,right=.7 in,top=.9in,bottom=.6in,%
            footskip=.25in]{geometry}
 
\usepackage{graphicx}
 \usepackage{amssymb}
 
\usepackage{lineno}

%\journal{Journal Name}
\begin{document}
%% Title, authors and addresses
\begin{center}
{\fontfamily{ppl}\selectfont{\large\textbf{Simulation of {\textit{Dictyostelium Discoideum}} Slugs Movement \\ with Kilobots}} }
\vspace{2mm}

{\fontfamily{qbk}\selectfont\textit{Mohammad Parhizkar \\ \small{February 2019} }}


%\vspace{2mm}
%\small{Information Systems\\Geneva School of Social Sciences \\Centre Universitaire d'Informatique\\\vspace{3mm}
%{\fontfamily{phv}\selectfont\small{Superviser: Prof. Giovanna Di Marzo Serugendo}}}
\end{center}
%% Text of abstract
\begin{abstract}
Understanding the collective behaviours in nature and its potential links to engineering the collective artificial behaviours in swarm robotics  have attracted the attention among researchers. They have various impacts on different domains such as cell-biology, cancer study, swarm of drones and unmanned robots. Since the cancer cells share similar collective behaviours, the biomedicine researchers look into different  examples from nature to design anti-cancer drugs to shrink tumours in human bodies. An interesting form of collective system is demonstrated by {\textit{Dictyostelium discoideum}}. 

   \end{abstract}

{\footnotesize\textit{Keywords:} {\textit{Dictyostelium discoideum}} - Kilobots - Swarm robotics - Multi-agent systems - Self-organising systems}

{\begin{center}\noindent\rule{14cm}{0.4pt}\end{center}}
%\noindent\makebox[\linewidth]{\rule{\paperwidth}{0.2pt}}
 %% ------------------------------------------------The first section ------------------------------------------------
\section{Problem}





 
 
\bibliographystyle{model1-num-names}
\bibliography{sample.bib}
 

\end{document}

%%
%% End of file `elsarticle-template-1-num.tex'.